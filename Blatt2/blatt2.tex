\documentclass{scrartcl}
\usepackage[utf8]{inputenc}
\usepackage[ngerman]{babel}
\usepackage{lmodern}
\usepackage{tikz}
\usepackage{amsmath}
\usepackage{amssymb}
\usepackage{geometry}
\usepackage{minted}
\usepackage{listings}
\usepackage{xcolor}
\usetikzlibrary{positioning, shapes}

\KOMAoptions{
    parskip=full
}

\tikzset{
    rel/.style={
        shape=diamond, draw=black,aspect=2, scale=1.5
    },
    att/.style={
        shape=ellipse, draw=black,inner sep=1, minimum width=0
    },
    ent/.style={
        shape=rectangle, draw=black, scale=1.7
    }
}

\newcommand{\DBTitel}[1]{
    \title{Datenbanken\\Blatt #1}
    \author{Youran Wang, 719511, RAS\\
        Paul Otto Reese, 723947, IT-Security\\
       Tim Hardow, 738356, Informatik\\Gruppe 4}
}


\let\nums\theenumi
\renewcommand{\theenumi}{\alph{enumi}}

\newcommand{\todo}{{\LARGE {\color{red}\textbf{TODO}}}}

\newcommand{\qed}[0]{
        {{\hfill $\square$}}
}

\lstset{
   backgroundcolor=\color{lightgray},
   extendedchars=true,
   basicstyle=\footnotesize\ttfamily,
   showstringspaces=false,
   language=SQL,
   showspaces=false,
   numbers=left,
   numberstyle=\footnotesize,
   numbersep=9pt,
   tabsize=2,
   breaklines=true,
   showtabs=false,
   captionpos=b,
   mathescape=false,
   stringstyle=\color{red}\ttfamily,
   literate={ö}{{\"o}}1
           {Ö}{{\"O}}1
           {ä}{{\"a}}1
           {Ä}{{\"A}}1
           {ü}{{\"u}}1
           {Ü}{{\"U}}1
           {ß}{{\ss}}1
}

\lstdefinelanguage{sql}{
	keywords={SELECT, FROM, WHERE, UPDATE, JOIN, CREATE, VIEW, INSERT, DOMAIN, CONSTRAINT, CHECK, VALUE},
	keywordstyle=\color{blue}\bfseries,
  	ndkeywords={AND, OR, AS, MIN, MAX, AVG, IN, STRING, INTEGER},
  	ndkeywordstyle=\color{red}\bfseries,
  	identifierstyle=\color{black},
  	comment=[l]{\#},
  	commentstyle=\color{purple}\ttfamily
}

\DBTitel{2}

\begin{document}

\maketitle

\section{Relationales Modell}
\subsection{Bei der relationalen Darstellung eines ER-Diagrams
werden nur Beziehungstypen als Relationsschemata dargestellt}
Falsch.

Es werden Beziehungen sowie Entitäten als Relationsschemata dargestellt.

\subsection{Bei der relationalen Darstellung von Beziehungstypen
werden Fremdschlüssel benutzt.}
Richtig.

Zum darstellen von Beziehungstypen werden Fremdschlüssel als Attribut in das Relationschemata einer Entität oder einer Beziehung eingefügt.

\section{Funktionale Abhängigkeiten}
\begin{center}
\begin{tabular}{|c|c|c|c|}
    \hline
    A & B & C & D\\
    \hline\hline
    $a_4$ & $b_2$ & $c_4$ & $d_3$\\
    \hline
    $a_1$ & $b_1$ & $c_1$ & $d_1$\\
    \hline
    $a_1$ & $b_1$ & $c_1$ & $d_2$\\
    \hline
    $a_2$ & $b_2$ & $c_3$ & $d_2$\\
    \hline
    $a_3$ & $b_2$ & $c_4$ & $d_3$ \\
    \hline
\end{tabular}    
\end{center}
\begin{enumerate}
\item $\{B\} \to \{A\}$

Falsch: $b_2$ kommt 3-mal vor, hat jedoch immer verschiedene a's in den entsprechenden Zeilen.
\item $\{B\} \to \{C, D\}$

Falsch: $b_2$ kommt 3-mal vor, hat jedoch immer verschiedene Kombinationen von c's und d's in den entsprechenden Zeilen.
\item $\{C, D\}$ ist ein Schlüssel

Falsch: $c_4$ und $d_3$ kommen in zwei Zeilen vor. Schlüssel müssen jedoch eindeutig sein.
\item $\{A, B, C, D\}$ ist ein Schlüssel

Richtig: Eine komplette Zeile muss einzigartig sein. Daher sind komplette Zeilen immer Schlüssel.
\item $\{B, D\}$ ist ein Kandidatenschlüssel.

Falsch: Da die Kombination $b_2$ und $d_3$ in zwei Zeilen vorkommt, ist $\{B, D\}$ nicht eindeutig und kann somit kein Kandidatenschlüssel sein.
\item $\{A, C, D\}$ ist ein Kandidatenschlüssel

Falsch: Es gilt $\{ A\} \to \{ C\}$ und somit ist $\{A, C, D\}$ nicht minimal und kann kein Kandidatenschlüssel sein.
\end{enumerate}

\section{Relationale Algebra}
Seien $r = (r_1,\ldots,r_n), s = (s_1, \ldots, s_n), q = (q_1, \ldots, q_n)$
\subsection{Es gilt die Distributivität von $\times$ über $\cup$:
$R \times (S \cup Q) = (R \times S) \cup (R \times Q)$}
Diese Aussage ist Richtig. Beweis:
\[(R \times S) \cup (R \times Q)\]
\[= \{(r,s)\ |\ r \in R \wedge s \in S\} \cup \{(r,q)\ |\ r \in R \wedge q \in Q\}\] 
\[ = \{x\ |\ [x \in \{(r,s)\ |\ r \in R \wedge s \in S\}] \vee [x \in \{(r,q)\ |\ r \in R \wedge q \in Q\}]\}\]
\[ = \{(r,x)\ |\ r \in R \wedge (x \in S \vee x \in Q)\}\]
\[= \{(r,x)\ |\ r \in R \wedge x \in S \cup Q\}\]
\[ = R \times (S \cup Q)\] \qed
\subsection{Selektionsoperationen kommutieren: $\sigma_{\theta_1}(\sigma_{\theta_2}(R))=\sigma_{\theta_2}(\sigma_{\theta_1}(R))$}
Diese Aussage ist auch Richtig. Es gilt der folgende Beweis:
\[\sigma_{\theta_1}(\sigma_{\theta_2}(R)) = \{r \in \sigma_{\theta_2}(R)\ |\ \theta_1(r)\} = \{r \in \{ r \in R\ |\ \theta_2(r)\}\ |\ \theta_1(r)\}\]\[ = \{r \in R\ |\ \theta_1(r) \wedge \theta_2(r)\}\]

\[\sigma_{\theta_2}(\sigma_{\theta_1}(R)) = \{r \in \sigma_{\theta_1}(R)\ |\ \theta_2(r)\} = \{r \in \{ r \in R\ |\ \theta_1(r)\}\ |\ \theta_2(r)\}\]\[ = \{r \in R\ |\ \theta_2(r) \wedge \theta_1(r)\}\]
Da \[\{r \in R\ |\ \theta_1(r) \wedge \theta_2(r)\} = \{r \in R\ |\ \theta_2(r) \wedge \theta_1(r)\}\] gilt \[\sigma_{\theta_1}(\sigma_{\theta_2}(R))=\sigma_{\theta_2}(\sigma_{\theta_1}(R)) \]\qed


\end{document}