\documentclass{scrartcl}
\usepackage[utf8]{inputenc}
\usepackage[ngerman]{babel}
\usepackage{lmodern}
\usepackage{tikz}
\usepackage{amsmath}
\usepackage{amssymb}
\usepackage{geometry}
\usepackage{minted}
\usepackage{listings}
\usepackage{xcolor}
\usetikzlibrary{positioning, shapes}

\KOMAoptions{
    parskip=full
}

\tikzset{
    rel/.style={
        shape=diamond, draw=black,aspect=2, scale=1.5
    },
    att/.style={
        shape=ellipse, draw=black,inner sep=1, minimum width=0
    },
    ent/.style={
        shape=rectangle, draw=black, scale=1.7
    }
}

\newcommand{\DBTitel}[1]{
    \title{Datenbanken\\Blatt #1}
    \author{Youran Wang, 719511, RAS\\
        Paul Otto Reese, 723947, IT-Security\\
       Tim Hardow, 738356, Informatik\\Gruppe 4}
}


\let\nums\theenumi
\renewcommand{\theenumi}{\alph{enumi}}

\newcommand{\todo}{{\LARGE {\color{red}\textbf{TODO}}}}

\newcommand{\qed}[0]{
        {{\hfill $\square$}}
}

\lstset{
   backgroundcolor=\color{lightgray},
   extendedchars=true,
   basicstyle=\footnotesize\ttfamily,
   showstringspaces=false,
   language=SQL,
   showspaces=false,
   numbers=left,
   numberstyle=\footnotesize,
   numbersep=9pt,
   tabsize=2,
   breaklines=true,
   showtabs=false,
   captionpos=b,
   mathescape=false,
   stringstyle=\color{red}\ttfamily,
   literate={ö}{{\"o}}1
           {Ö}{{\"O}}1
           {ä}{{\"a}}1
           {Ä}{{\"A}}1
           {ü}{{\"u}}1
           {Ü}{{\"U}}1
           {ß}{{\ss}}1
}

\lstdefinelanguage{sql}{
	keywords={SELECT, FROM, WHERE, UPDATE, JOIN, CREATE, VIEW, INSERT, DOMAIN, CONSTRAINT, CHECK, VALUE},
	keywordstyle=\color{blue}\bfseries,
  	ndkeywords={AND, OR, AS, MIN, MAX, AVG, IN, STRING, INTEGER},
  	ndkeywordstyle=\color{red}\bfseries,
  	identifierstyle=\color{black},
  	comment=[l]{\#},
  	commentstyle=\color{purple}\ttfamily
}

\DBTitel{7}

\begin{document}

\maketitle

\section{Verständnisfragen zu Indexen}
\begin{enumerate}
    \item \textbf{Ein ISAM-Index verweist auf den höchsten Schlüssel eines Datenblockes der Hauptdatei, die aufsteigend sortiert ist.}
    
    Falsch, denn ein ISAM-Index gibt immer die ``Zwischenräume'' zwischen zwei Datenblöcken an. Das bedeutet er ist die obere (exklusive) Grenze für den linken und die untere (inklusive) Grenze für den rechten Datenblock. 
    \item \textbf{Bei der Verwendung einer indexsequentiellen Zugriffsmethode müssen stark wachsende oder sich verändernde Dateien zeitaufwändig regelmäßig reorganisiert werden.}
    
    Stimmt, da der ISAM-Index degeneriert und bei stark wachsenden Datenbanken sehr häufig Überlaufseiten erzeugt werden und der Index folglich reorganisiert werden muss.
    \item \textbf{Eine Datei kann mehrere verschiedene Indizes haben, aber nicht mehr als einen geclusterten Index.}
    
    Diese Aussage stimmt, da mehrere Indizes auf einen geclusterten Index zeigen können, aber eine Datei nur einen geclusterten Index haben kann, da dieser eindeutig einer Datei zuordenbar sein muss.
    \item \textbf{Um einen zunehmenden Aufwand zur Kollisionsauflösung zu vermeiden, wird beim Dynamischen Hashing die Hashtabelle bei Bedarf vergrößert.}
    
    Diese Aussage stimmt, da bei zu kleinem n der Aufwand zur Kollisionsauflösung sehr groß wird wird und durch dynamisches Hashing hier Abhilfe geschaffen wird.
    \item \textbf{Bei B$^+$-Bäumen werden auch in inneren Knoten tatsächliche Daten abgespeichert.}
    
    In inneren Knoten werden lediglich Indizes der darunter liegenden Blöcke abgespeichert und keine ``tatsächlichen'' Daten. Somit ist die Aussage falsch.
    \item \textbf{B$^+$-Indexierung ist für Bereichsanfragen (in Datenbanksystem) optimal geeignet.}
    
    Diese Aussage stimmt nur für geclusterte B$^+$ Bäume, da man hier einfach sequentiell lesen kann. Für allgemeine B$^+$ Bäume gilt das aber nicht, da die Bereiche hier stark fragmentiert vorliegen könnten.
\end{enumerate}

\end{document}