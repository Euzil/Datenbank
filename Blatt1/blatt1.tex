\documentclass{scrartcl}
\usepackage[utf8]{inputenc}
\usepackage[ngerman]{babel}
\usepackage{lmodern}
\usepackage{tikz}
\usepackage{amsmath}
\usepackage{amssymb}
\usepackage{geometry}
\usepackage{minted}
\usepackage{listings}
\usepackage{xcolor}
\usetikzlibrary{positioning, shapes}

\KOMAoptions{
    parskip=full
}

\tikzset{
    rel/.style={
        shape=diamond, draw=black,aspect=2, scale=1.5
    },
    att/.style={
        shape=ellipse, draw=black,inner sep=1, minimum width=0
    },
    ent/.style={
        shape=rectangle, draw=black, scale=1.7
    }
}

\newcommand{\DBTitel}[1]{
    \title{Datenbanken\\Blatt #1}
    \author{Youran Wang, 719511, RAS\\
        Paul Otto Reese, 723947, IT-Security\\
       Tim Hardow, 738356, Informatik\\Gruppe 4}
}


\let\nums\theenumi
\renewcommand{\theenumi}{\alph{enumi}}

\newcommand{\todo}{{\LARGE {\color{red}\textbf{TODO}}}}

\newcommand{\qed}[0]{
        {{\hfill $\square$}}
}

\lstset{
   backgroundcolor=\color{lightgray},
   extendedchars=true,
   basicstyle=\footnotesize\ttfamily,
   showstringspaces=false,
   language=SQL,
   showspaces=false,
   numbers=left,
   numberstyle=\footnotesize,
   numbersep=9pt,
   tabsize=2,
   breaklines=true,
   showtabs=false,
   captionpos=b,
   mathescape=false,
   stringstyle=\color{red}\ttfamily,
   literate={ö}{{\"o}}1
           {Ö}{{\"O}}1
           {ä}{{\"a}}1
           {Ä}{{\"A}}1
           {ü}{{\"u}}1
           {Ü}{{\"U}}1
           {ß}{{\ss}}1
}

\lstdefinelanguage{sql}{
	keywords={SELECT, FROM, WHERE, UPDATE, JOIN, CREATE, VIEW, INSERT, DOMAIN, CONSTRAINT, CHECK, VALUE},
	keywordstyle=\color{blue}\bfseries,
  	ndkeywords={AND, OR, AS, MIN, MAX, AVG, IN, STRING, INTEGER},
  	ndkeywordstyle=\color{red}\bfseries,
  	identifierstyle=\color{black},
  	comment=[l]{\#},
  	commentstyle=\color{purple}\ttfamily
}

\DBTitel{1}

\begin{document}

\maketitle

\section{Verständnisfragen zum ER-Modell}
\begin{enumerate}
    \item Das ist falsch. Ein Dieselfahrzeug ist keine Generalisierung sondern eine Spezialisierung eines Fahrzeug. Denn nicht jedes Fahrzeug ist ein Dieselfahzeug. 
    \item Das ist falsch. Der Entitätstyp Mader stünde in einer is-a Beziehung zu Fleischfresser, könnte jedoch auch in der Beziehung wird-gefressen zum Fleischfresser stehen, da Mader z.B. von Greifvögeln gefressen werden.
    \item Das ist wahr. Eine Superklasse ist immer eine Teilmenge der darunterliegenden Klasse. Da eine Menge stets Teilmenge von sich selbst ist, ist auch eine Klasse stets ihre eigene Superklasse.
    \item Das ist wahr. Analog zu c.
    \item Das ist wahr. Als Beispiel nehme man das ER-Modell aus Aufgabe 2. Man könnte von der Relation fährt eine Beziehung Sonderfahrt ableiten, welche zusätzlich noch das Attribut Grund enthält. Diese würde dann, wie bei Entitäten, das Fahrdatum erben.
    \item Das ist wahr. Es können auch mehrere Attribute einen Schlüssel ergeben.
    \item Das ist falsch. Wenn die Entität Airbag in Teil-von Beziehung zu der Entität Auto steht, so steht er nicht automatisch in einer is-a Relation zum Auto.
\end{enumerate}

\section{ER-Modelle versprachlichen}

\renewcommand{\theenumi}{\nums}
\begin{itemize}
    \item Es gibt 2 Entitäten: Lockführer und Fahrt.
    \item Lockführer wird beschrieben durch:
    \begin{enumerate}
        \item eine Lockführernummer
        \item eine Lockführername
    \end{enumerate}
    Die Lockführernummer ist der Primärschlüssel des Lockfürer.
    \item Fahrt wird beschrieben durch:
    \begin{enumerate}
        \item der Startort
        \item der Zielort
        \item einen Fahrtcode
    \end{enumerate}
    Der Fahrtcode ist der Primärschlüssel des Entitätstyps Fahrt.
    \item Zwischen Lockfürer und Fahrt ist eine Assziation “fährt”, die mit Fahrtdatum beschrieben wird. Lockführernummer und Fahrtcode sind Fremdschlüssel bei “fährt”.
    \item Funktionalitätsangaben: Einem Lockfürer können n Fahrten zugeordnet werden, einer Fahrt jedoch nur einem Lockführer. Folglich kann ein Lockführer mehrere Fahren machen, aber eine Fahrt kann nur von einem Lockführer gefahren werden. 
\end{itemize}
\begin{tikzpicture}
\node[ent] at (0,0) (lf) {Lockführer};
\node[att, above=of lf] (lfnr) {\underline{Lockführernummer}};
\node[att,below=of lf] (lfnm) {Lockführername};
\node[rel, right=of lf] (f) {fährt};
\node[att,below=of f] (fdt) {Fahrtdatum};
\node[ent, right=of f] (frt) {Fahrt};
\node[att, above=of frt] (frtcd) {\underline{Fahrtcode}};
\node[att, right=of frt] (frtso) {Startort};
\node[att, below right=of frt] (frtzo) {Zielort};
\node[rel, below=of frt] (fm) {fährt mit};
\node[ent, below=of fm] (fg) {Fahrgast};

\path[-]
    (lf) edge (lfnr)
    (lf) edge (lfnm)
    (lf) edge node[above] {1} (f)
    (f) edge node[above] {N} (frt)
    (f) edge (fdt)
    (frt) edge (frtcd)
    (frt) edge (frtso)
    (frt) edge (frtzo)
    (frt) edge node[right] {1} (fm)
    (fg) edge  node[right] {N} (fm);
\end{tikzpicture}

\end{document}